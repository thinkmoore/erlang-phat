\documentclass[10pt,letter]{article}

\usepackage{amsmath}
\usepackage{amssymb}
% packages that allow mathematical formatting

\usepackage{graphicx}
% package that allows you to include graphics

\usepackage{setspace}
% package that allows you to change spacing

\usepackage{cite}
% for BibTeX

\usepackage{algpseudocode}
\usepackage{algorithmicx}
\usepackage{algorithm}

\usepackage[sc,osf]{mathpazo}
\linespread{1.05}         % Palatino needs more leading
\usepackage[T1]{fontenc}

\fontencoding{T1}
\fontfamily{ppl}
\fontseries{m}
\fontshape{n}
\fontsize{10}{13}
% Set font size. The first parameter is the font size to switch to; the second
% is the \baselineskip to use. The unit of both parameters defaults to pt. A
% rule of thumb is that the baselineskip should be 1.2 times the font size.
\selectfont

\usepackage{parskip}
% Use the style of having no indentation with a space between paragraphs.

\usepackage{enumitem}
% Resume an enumerated list, continuing the old numbering, after some
% intervening text.

% \usepackage{fullpage}
% package that specifies normal margins

\usepackage{titling}
% move the title up a bit, wastes less space
\setlength{\droptitle}{-5em}


\usepackage[usenames,dvipsnames]{color}
% for \color used in listings
\usepackage{textcomp}
% for upquote used in listings
\usepackage{listings}
\lstset{
    breakatwhitespace, % somehow prevents lstinline from line splitting
    tabsize=2,
    rulecolor=,
    basicstyle=\footnotesize,
    upquote=true,
    aboveskip={1.5\baselineskip},
    columns=fixed,
    extendedchars=true,
    breaklines=true,
    prebreak = \raisebox{0ex}[0ex][0ex]{\ensuremath{\hookleftarrow}},
    frame=single,
    showtabs=false,
    showspaces=false,
    showstringspaces=false,
    keywordstyle=\color{Purple},
    identifierstyle=\color{Black},
    commentstyle=\color{BrickRed},
    stringstyle=\color{RubineRed},
    captionpos=b,
}
\lstloadlanguages{erlang}
\lstset{language=erlang}

\usepackage{tikz}
\usetikzlibrary{graphdrawing}

%%%%%%%%%%%%%%%%%%%%%%%%%%%%%%%%%%%%%%%%%%%%%%%%%%%%%%%%%%%%%%%%%%%%%%%%%%%%%%%
%% Macros

\newcommand{\chubby}[0]{\textsc{Chubby}}
\newcommand{\phat}[0]{\textsc{Phat}}
\newcommand{\phatraid}[0]{\textsc{PhatRaid}}
\newcommand{\raid}[1]{\textsc{RAID #1}}
\newcommand{\paxos}[0]{\textsc{Paxos}}

%%%%%%%%%%%%%%%%%%%%%%%%%%%%%%%%%%%%%%%%%%%%%%%%%%%%%%%%%%%%%%%%%%%%%%%%%%%%%%%

\begin{document}

\title{PHAT RAID, yo}
\author{Andrew Johnson \and Daniel King \and Lucas Waye \and Scott Moore}
\date{May 13, 2014}

\maketitle

Distributed file stores, like Google's \chubby{} and CS260r's \phat{}, send all
files through a single master node which consistently replicates the file on
many slaves. In this scheme, the master node's throughput is a bottleneck for
storing large files. We present \phatraid{} which mitigates this bottleneck by
partitioning files across many \paxos{} groups which form a \raid{1} group.

\section{Introduction}

\section{Implementation}

\subsection{System Organization}

\phatraid{} is implemented in Erlang, reusing our previous implementation of
\phat{}. The implementation is divided into three domains: the server cluster,
the client, and the raid-client. The server code implements a VR-like consensus
algorithm as well as simple file system. The client code exposes a file system
API which hides the communication bookkeeping. The raid-client provides a file
system API which transparently partitions files and reassembles them as they are
sent and received from the server cluster.



\subsection{Erlang and Behaviors}

In Erlang, the basic unit of concurrency is a \emph{process} and an Erlang VM is
known as a \emph{node}. A process may communicate with other processes on its
node as well as processes on other nodes. Multiple nodes may be on the same
machine or distributed across many machines.

The server cluster is implemented as a collection of nodes running the same
server code. One of these nodes is designated the master node. The other nodes
are known as replicas.

The master and each replica has four components: the supervisor, the
(unfortunately named) server, VR, and the file system.

\begin{itemize}
\item the supervisor -- if any of the other three processes dies, it shuts all
  of them down and restarts all of them
\item the server -- if the node is the master, it passes messages on to VR; if
  the node is a replica, it redirects clients to the master
\item VR -- maintains a log of opaque messages, achieving consensus with other
  nodes in the cluster via the Viewstamped Replication protocol
\item the file system -- a simple hierarchical file system with locks
\end{itemize}

The four components are implemented using Erlang behaviors. An Erlang behavior
is a framework which implements common patterns like servers and finite state
machines. In particular, the supervisor uses the \texttt{supervisor} behavior,
the server and file system both use the \texttt{gen\_server} behavior, and VR
uses the \texttt{gen\_fsm} behavior.

\begin{itemize}
\item The \texttt{supervisor} behavior provides a framework for restarting
  failed processes.
\item The \texttt{gen\_server} behavior provides a framework for many-client
  single-server interactions. We implemented functions to process messages
  formatted as Erlang tuples. The \texttt{gen\_server} behavior handles
  low-level socket interaction.
\item The \texttt{gen\_fsm} behavior generalizes the \texttt{gen\_server}
  behavior by permitting the server to have a finite number of states. Each
  state has a set of functions with which to process messages. This behavior can
  be seen as an ad hoc polymorphic variant of the \texttt{gen\_server} behavior.
\end{itemize}

Additionally, the \texttt{gen\_server} and \texttt{gen\_fsm} behaviors allow
the programmer to specify an arbitrary state value which will be passed around
\`{a} la functional reactive programming.

\subsubsection{The \texttt{supervisor} Behavior}

\lstinputlisting[firstline=6,lastline=18,float,
                 caption=The \texttt{supervisor} Behavior --- \texttt{phat.erl},
                 label=lst:supervisor,
                 numbers=left, firstnumber=6]
                {../phat.erl}

Listing \ref{lst:supervisor} is taken from our supervisor code. When a
supervisor is started, the \texttt{supervisor} behavior looks for a unary
procedure called \lstinline!init!. The \lstinline!init! procedure returns a
child specification which is a pair of a restart specification and a list of
children. In Listing \ref{lst:supervisor}, the restart specification,
\lstinline!{one_for_all,1,5}! specifies that if any one node dies, all nodes
should be restarted, unless more than one failure has occurred in the past five
seconds. If more than one failure occurs in five seconds, the supervisor kills
all the children and shuts itself down as well.

Each child in the list of children is a 6-tuple consisting of: child name,
initial procedure call, child transience, shutdown style, child type, necessary
modules.

\begin{itemize}
\item The initial procedure call is a 3-tuple of module, procedure name and
  argument list.
\item Every child's transience is specified as \lstinline!permanent!, meaning
  they should be restarted.
\item Every child has shutdown style \lstinline!1! meaning they will first be
  asked to terminate and one second later they will be forcibly terminated.
\item All of our children are workers, which are distinguished from
  sub-supervisors.
\item The necessary modules are dynamically loaded when the child processes is
  initialized, there is a one-to-one mapping between our children and code
  modules.
\end{itemize}

\subsubsection{The \texttt{gen\_fsm} Behavior}

\begin{lstlisting}[float,caption=The \texttt{gen\_fsm} Behavior --- \texttt{vr.erl},
                   label=lst:genfsm, numbers=left, firstnumber=176]
replica( {prepare, MasterViewNumber, _, _, _, _, _}
       , State = #{ viewNumber := ViewNumber})
    when MasterViewNumber > ViewNumber ->
    ?debugFmt("my view is out of date, need to recover~n", []),
    startRecovery(State);

replica( {prepare, MasterViewNumber, _, _, _, _, _}
       , State = #{ timeout := Timeout, viewNumber := ViewNumber})
    when MasterViewNumber < ViewNumber ->
    ?debugFmt("ignoring prepare from old view~n", []),
    {next_state, replica, State, Timeout};

replica( { prepare, _, Op, OpNumber
         , MasterCommitNumber, Client, RequestNum}
       , State = #{ prepareBuffer := PrepareBuffer }) ->
    Message = {OpNumber, Op, Client, RequestNum},
    NewPrepareBuffer = lists:sort([Message|PrepareBuffer]),
    processPrepareOrCommit( OpNumber
                          , MasterCommitNumber
                          , NewPrepareBuffer
                          , State
                          );
\end{lstlisting}

\begin{lstlisting}[float,caption={Processing the VR Message Queue, Part 1 --- \texttt{vr.erl}},
                   label=lst:processing1, numbers=left, firstnumber=375]
processPrepareOrCommit( OpNumber, MasterCommitNumber, NewPrepareBuffer
                      , State = #{ timeout := Timeout
                                 , commitNumber := CommitNumber
                                 , masterNode := MasterNode
                                 , myNode := MyNode
                                 , viewNumber := ViewNumber
                                 , allNodes := Nodes})
  when CommitNumber > MasterCommitNumber ->
    NewViewNumber = ViewNumber + 1,
    NewMaster = chooseMaster(State, NewViewNumber),
    sendToReplicas( MasterNode
                  , Nodes
                  , {startViewChange, NewViewNumber, MyNode}
                  ),
    { next_state
    , viewChange
    , State#{ viewNumber := NewViewNumbern
            , masterNode := NewMaster }
    , Timeout
    };
\end{lstlisting}
\begin{lstlisting}[float,caption={Processing the VR Message Queue, Part 2 --- \texttt{vr.erl}},
                   label=lst:processing2, numbers=left, firstnumber=397]
processPrepareOrCommit( OpNumber, MasterCommitNumber, NewPrepareBuffer
                      , #{ timeout := Timeout } = State) ->
    AfterBuffer =
      processBuffer( State#{ prepareBuffer := NewPrepareBuffer }
                   , NewPrepareBuffer
                   , MasterCommitNumber
                   ),
    AfterLog = processLog(AfterBuffer, MasterCommitNumber),
    #{ masterNode := MasterNode
     , viewNumber := ViewNumber
     , commitNumber := CommitNumber
     , myNode := MyNode } = AfterLog,
    if
        CommitNumber < MasterCommitNumber ->
            startRecovery(State);
        true ->
            sendToMaster( MasterNode
                        , {prepareOk, ViewNumber, OpNumber, MyNode}
                        ),
            {next_state, replica, AfterLog, Timeout}
    end.
\end{lstlisting}

A state in the \texttt{gen\_fsm} behavior manifests as a procedure which accepts
two arguments: the incoming message and the store\footnote{Mutable state is
  implemented as a pure store value which is passed around by the behavior}. The
return value of a state-procedure is usually the 4-tuple
\lstinline!{next_state,NextState,NewStore,Timeout}!. The \texttt{gen\_fsm}
behavior will transition to \lstinline!NextState! and set the store to
\lstinline!NewStore!. When a new message is received the \texttt{gen\_fsm}
behavior will invoke the \lstinline!NextState! procedure with the new message
and the \lstinline!NewStore!.

We implemented Viewstamped Replication\cite{liskov2012viewstamped} to achieve
consensus in the file system. Listing \ref{lst:genfsm} defines how to handle
prepare messages sent to a node in the replica state. It is piecewise defined by
pattern matching on the arguments. The first two definition clauses use the
\lstinline!when! keyword to specify arbitrary, required relationships between
the matched variables.

The final definition clause triggers when the replica receives a message in the
current view. The replica adds the message to a sorted queue and calls a
processing procedure. The processing procedure is defined separately in Listing
\ref{lst:processing1} and Listing \ref{lst:processing2}.

Listing \ref{lst:processing1} handles a message from an out-of-date master,
i.e., his commit number is older than the replica's commit number. The replica
first proposes a view change to the cluster. Afterwards, it changes its
\texttt{gen\_fsm} state to \lstinline!viewChange! and updates the
store\footnote{The store is called \lstinline!State! in the listings} with the
new view number and the new master.

Listing \ref{lst:processing2} handles messages from the current master. The
\lstinline!prepareBuffer! stores prepare messages that arrived out of order. In
particular, if the log ends at operation number $n$ and message $n+1$ is dropped
by the network, all subsequent messages will be buffered. The buffered messages
will not be added to the log until message $n+1$ is received. The call to
\lstinline!processBuffer! moves messages to the log, if all previous messages
have now been received. The call to \lstinline!processLog! commits new messages
in the log which have been committed by the master node. Lines 405 to 408
destructure \lstinline!AfterLog! and bind variables for later use. The final
\lstinline!if! statement responds to the master unless the master has committed
messages we haven't received.\footnote{This can happen if the network drops a message whose
  operation number lies between the replica's last commit number and the
  master's last commit number.}

\section{Performance}

\section{Conclusion}

\bibliographystyle{plain}
\bibliography{paper}

\end{document}
